% !TEX root = ../report.tex

\chapter{Evaluation}
\label{chap:resulteval}
\minitoc

%TODO intro

\clearpage

% !TEX root = ../../report.tex

\section{Experimental Results}

The following section will present our experimental results, dicuss our findings and
discuss whether or not our results supports our hypotheses.

\section{Binary Purchase Only Dataset}

Attempting to do collaborative filtering, let alone do recommendations with only 466 users, 1188 items and
an average of 1.34 ratings per items is no easy task using traditional recommender systems. However, these numbers
are meant as a baseline to show what one could expect when only looking at purchase data.

\begin{table}[H]
    \centering
    \begin{tabular}{*{5}l}
    \toprule
    Model 			&	AUC			&	$MAP@20$ \\ \midrule
    \rowcolor{Gray}
    Most Popular	&	0.30314		&	0.00000	\\
    ItemBasedKNN	&	0.52366		&	0.00000	\\
    UserBasedKNN	&	0.52189		&	0.00000	\\
    BPR-MF			&	0.34490		&	0.00000	\\
    \bottomrule
    \end{tabular}
\caption[Experimental Results - Purchase Only Dataset]{Experimental Results for the purchase only dataset using random splits. The results are averaged over 5 runs}
\end{table}

To our surprise our Most Popular recommender performed much worse than random guessing. But since the average item
have been given on average 1.34 ratings one could not expect to much since the ratings are evenly distribution over
all the items. Both item-based and user-based collaborative filtering performed marginally better than random guessing,
while BPR-MF performed much worse than random guessing. The $MAP@20$ scores for all recommenders are to small to be shown
using 5 descimals, meaning that close to zero relevant items were retrieved in the top 20 lists.

\section{Multiple Events Datasets}

When incorporating both clicks and likes in addition to purchases we ended up with a total of 1,511 users, 5,855 items
and an average of 4.76 ratings per item.

Here each event type is seen as equally important and is not weighted differently.
These numbers are meant as a baseline to show what is possible using binary positive feedback only recommendation methods.
Our goal is therefore to prove that these results can be beaten using our implicit ratings.

This subsection will present the experimental results for the SoBazaar dataset using implicit ratings. First we present
our results using both time-based and random splits testing different implicit rating mapping functions on different recommendation methods.

\subsection{Time-based splits}

As we have access to timestamps we decided to split the dataset on time, making the evaluation more \textit{realistic}. However,
there seem to have been particularly few purchase activity over the last two months. We first attempted to evaluate the recommenders
using a 90:10 timesplit, making predictions for the events between 7. may - 19. may. However, there have only been recorded two purchase or 0.07\% of
all events in the last twelve day period, making it a bad period to use for evaluation. We did therefor not run multiple runs on this
data, as we thought it more beneficial to get a better split to evaluate on. The results we did get can be found in the appendix \ref{90:10-timesplits}
We instead opted to use a 80:20 timesplit, predicting the events between 16. April - 22. May. Our results are shown in following table:

\begin{table}[H]
\centering
\resizebox{
\columnwidth}{!}{
\begin{tabular}{*{19}l}
\toprule
Model & AUC &	$MAP@20$ &	$T_{click}$ &	$T_{want}$ &	$T_{purchase}$ &	$P_{click}$&	$P_{want}$ &	$P_{purchase}$ &	$R_{click}$ &	$R_{want}$ &	$R_{purchase}$ &	$MAP@20_{click}$ &	$MAP@20_{want}$	& $MAP@20_{purchase}$ &	 \\
\midrule
\rowcolor{Gray}
Most Popular 		 		&	0.613218 &	0.004856 &	3038 &	2328 &	36 &	28 &	10 &	0 &	0.009217 &	0.004296 &	0 		 &	0.004622 &	0.003046 &	0 		 &	 \\
\rowcolor{Gray}
ItemBasedKNN k=100			& 	0.464325 &	0.002638 &	3038 &	2328 &	36 &	14 &	7  &	1 &	0.004608 &	0.003007 &	0.027778 &	0.002576 &	0.000623 &	0.002778 &	 \\
\rowcolor{Gray}
UserBasedKNN k=200  		&	0.585442 &	0.004358 &	3038 &	2328 &	36 &	17 &	14 &	3 &	0.005596 &	0.006014 &	0.083333 &	0.002927 &	0.007883 &	0.053241 &	 \\
\rowcolor{Gray}
BPR-MF	 					&	0.601683 &	0.004902 &	3038 &	2328 &	36 &	21.8 &	9 &	  0.8 &	0.007176 &	0.003866 &	0.022222 &	0.004574 &	0.002854 &	0.004762 &	\\
\rowcolor{Gray}
BPR-Linear					&	0.559033 &	0.000443 &	3038 &	2328 &	36 &	2 	&	2 &		0 &	0.000658 &	0.000859 &	0 		 &	0.000136 &	0.000709 &	0 		 &	\\
ALS-WR Count Linear         &   0.628547 &  0.006958 &  3037 &  2328 &  36 & 19.25 & 22.25 & 	3 & 0.006338 &  0.009558 &  0.083333 &  0.005049 &  0.009524 &  0.062978 &   \\
ALS-WR Popularity Linear    &   0.625682 &  0.001705 &  3037 &  2328 &  36 & 13.5  & 12.75 & 	3 & 0.004445 &  0.005477 &  0.083333 &  0.000739 &  0.003915 &  0.014694 &   \\
ALS-WR Recentness Linear    &   0.631948 &  0.006444 &  3037 &  2328 &  36 & 19    & 17.5  & 	3 & 0.006256 &  0.007517 &  0.083333 &  0.004529 &  0.010394 &  0.073505 &   \\
ALS-WR Price Linear         &   0.611358 &  0.003687 &  3037 &  2328 &  36 & 18.75 & 13    &    4 & 0.006174 &  0.005584 &  0.111111 &  0.002123 &  0.007305 &  0.038556 &   \\
ALS-WR Blend 1              &   0.623146 &  0.005666 &  3037 &  2328 &  36 & 19.25 & 14.5  &  	3 & 0.006338 &  0.006229 &  0.083333 &  0.004144 &  0.007086 &  0.059737 &   \\
\bottomrule
\end{tabular}
}
\caption{Experimental results time-based splits 80:20 (16. April - 19. May)}
\end{table}

%TODO - Generate Averages!!! (BPR-MF, BPRLinear, UserBased '50, 80', ItemBased 50, 80)
%TODO - Test ItemBased with Cosine Similarity '50, 80, 100' and ItemBased with log likelihood

Again we see that the event distribution if far from optimal as it contains only 0.6\% purchases. However, further increasing the size of the split did not improve this ratio noticeably.
Again we can see that we were not able to beat the most-popular recommender significantly using any recommendation methods or parameter settings. The only area where the most-popular recommender
was beaten was on recommending liked and purchases items.

%TODO - Average Results! (Atleast 3 runs of each)
%TODO - Test more ALS-WR parameters (Atleast 100, 100, 15, true, 10), can we beat binary?...
%TODO - Test userbased 150 neighbors
%TODO - Compare with binary results and other blends (Only when doing this you do not have to run all implicit rating files, ONLY THE BLEND!)
%Blend 1 - Count linear + Price linear + Popularity linear + Recentness linear
%Blend 2 - Remove recentness linear
%Blend 3 - Recentness linear + Recentness sigmoid

As you can see from the table, our Item-Average results are varying strongly. The blend gives the best recommendations retrieving the most items.
There is no dithering implemented in the mahout recommender, giving us non decimal numbers. ALS-WR is superior to all the other methods over all metrics.
It is was a pleasant surprise to see that both our implicit rating functions based on recency have the best performance, since this is a time-split.

Compared to the binary only ratings the results can be considered marginally better. Both in terms of AUC, MAP\@20 and the overall recall scores.


\subsection{Random splits}

Since we were not completely satisfied with the event distribution on our time-splits we decided to also evaluate the recommenders on random dataset splits.

The following results are generated using a random dataset splitter. This was performed due to the low amounts of purchase activity the last couple of months to get an even better
overview of the systems performance.

\begin{table}[H]
\centering
\resizebox{
\columnwidth}{!}{
\begin{tabular}{*{19}l}
\toprule
Model & AUC &	$MAP@20$ &	$T_{click}$ &	$T_{want}$ &	$T_{purchase}$ &	$P_{click}$&	$P_{want}$ & $P_{purchase}$ &	$R_{click}$ &	$R_{want}$ &	$R_{purchase}$ &	$MAP@20_{click}$ &	$MAP@20_{want}$	& $MAP@20_{purchase}$ &	 \\
\midrule
%Binary Methods
\rowcolor{Gray}
Most-Popular    			&  0.751516     & 0.013236  &   1323    &   1235    &   135     &   63      &   40.75   & 3.5     &   0.0476135   &   0.03300275  &   0.0270445   &   0.0147725   & 0.0111475 &   0.0098995 & \\
\rowcolor{Gray}
ItemBased Log	   			&  0.690723     & 0.002094  &   1353.9  &   1250.5  &   138     &   8.8     &   4.6     & 1.47    &   0.006488    &   0.003672    &   0.007252    &   0.001470    & 0.002043  &   0.000618  & \\
\rowcolor{Gray}
ItemBasedKNN k=120  		&  0.7357796    & 0.0069094 &   1295.8  &   1231.4  &   146.8   & 	37.6    &   53      & 7.2     &   0.0290256   &  0.0431066    &  0.0487262    &  0.0044356    &  0.0097038&   0.0060544 & \\
\rowcolor{Gray}
UserBasedKNN K = 200 cos	&  0.7926842	& 0.0317002	&	1282.2	&   1186.2 &	136.4	&	102.2	&	101.4	& 12.6    &	  0.0797042	  &	0.0854958	  &	0.471684	  &	0.128002	  &	0.182436  &	0.0349802	&	\\
\rowcolor{Gray}
BPR-MF          			&  0.7286434    & 0.0099586 &   1332.8  &   1233.8  &   135.4   &   51.4    &   31      & 5.2     &   0.0385224   &   0.025142    &   0.0385042   &   0.0120152   & 0.0058942 &   0.0133786 & \\
\rowcolor{Gray}
BPRLinear       			&  0.7097516    & 0.0070648 &   1344.3  &   1241.9  &   141.3   &   27.4    &   25.1    & 2.5     &   0.0203941   &   0.0202133   &   0.017821    &   0.006298    & 0.0075964 &   0.0022692 & \\
%ItemBased
IBCF Price linear			&	0.680758 &	0.004616 &	1332.8 &	1245.7 &	137.15  &	21.1   		&	17.6    &	5.63   &	0.015859 &	0.014138 &	0.041189 &	0.003722 &	0.005    &	0.008681 &	 \\
IBCF Popularity Linear		&	0.554495 &	0.001209 &	1335.7 &	1230.1 &	149.9 	&	0.3    		&	6 	    &	5.95   &	0.000198 &	0.004871 &	0.038575 &	0.000028 &	0.002344 &	0.006537 &	 \\
IBCF Blend				    &	0.681773 &	0.006226 &	1328.5 &	1245   &	147.5   &	20.5        &	18.5    &	4.5    &	0.015464 &	0.014836 &	0.030416 &	0.004825 &	0.006603 &	0.009863 &	 \\
%ALS-WR 100 100 15 true 10
ALS-WR Price linear			&	0.797892 &  0.034154  &	1340.5	&	1250.8	&	138.0	&	108.3		&	121.4	&	15.0   &	0.080842 &	0.09701	 &	0.108833 &	0.028443 &	0.035474 &	0.033493 & 	\\
ALS-WR Popularity Linear 	&	0.816561 &	0.023916 &	1340.5	&	1246.25 &	139   	&	86.5		&	98.25   &	8.5    &	0.064534 &	0.078812 &	0.060836 &	0.018396 &	0.031169 &	0.01758 &	\\
ALS-WR Blend 				&	0.800422 &	0.034519 &	1344.75 &	1246.5  &	134.5 	&	104.5 		&	120.75  &	13.5   &	0.077686 &	0.097038 &	0.100486 &	0.026307 &	0.041865 &	0.025649 &	 \\
ALS-WR Blend				&	0.808125 &	0.035802 &	1305.5 &	1221.5 &	150.5   &	98.5 		&	123.5   &	16 	   &	0.075447 &	0.101033 &	0.106291 &	0.027193 &	0.042275 &	0.05209 &	 \\
\bottomrule
\end{tabular}
}
\caption{Experimental results random splits 90:10 - The Results are averaged over 5-10 runs}
\end{table}

%TODO - If time, play with BPR-MF Parameters
%TODO - Makes no sense to look at recentness here, make other blends
%TODO - If time, more tuning of ALS-WR parameters
%TODO - Test two or three more blends
%TODO - Compare with binary results, do we see any improvements?


It was interesting to see that the performance of ItemBasedKNN actually started to decrease when looking at more than 50 neighbors, while UserBasedKNN's performance kept going up or stayed mostly the same when adding 
neighbors. UserBasedKNN with between 50-200 neighbors seem to give the best recommendation quality of all the methods tested, and they different is fairly significant.


As you can see the random splits significantly improved our event distribution giving us approximately 5\% purchase events for each run.
As you can see from the table, ItemBasedKNN using cosine similarity gives us the best results for our binary dataset. It has
the best AUC and retrieved 9.67\% of all wanted items and 9.85\% of all purchases in addition to having the best $MAP@20_{purchase}$ score.











\subsection{Cold-Start Results}

The following subsections will present our results from our cold-start evaluation runs, first measuring the native cold-start performance
of the different methods, then we apply our cold-start solution, filterbots, and see if the recommendation quality improves.
Running cold-start experiments costly in the sense that we need to train 9 different models for each implicit rating function and model.
In addition we also have $2^5$ possible filterbots combinations, and the option of using different parameters for each method.

As ALS-WR and ItemBasedCF gave us the best results on our random split we chose to only apply filterbots to these models.



\subsection{Time-based splits}


\begin{table}[H]
\centering
\resizebox{\columnwidth}{!} & \multicolumn{1}{c}{40\%} & \multicolumn{1}{c}{75\%} & \multicolumn{1}{c}{10\%} & \multicolumn{1}{c}{40\%} & \multicolumn{1}{c}{75\%} & \multicolumn{1}{c}{10\%} &
\multicolumn{1}{c}{40\%} & \multicolumn{1}{c}{75\%} & \multicolumn{1}{c}{10\%} & \multicolumn{1}{c}{40\%} & \multicolumn{1}{c}{75\%} & \multicolumn{1}{c}{10\%} & \multicolumn{1}{c}{40\%} &
\multicolumn{1}{c}{75\%} & 	\\ \midrule 


%PRE FILTERBOTS
\rowcolor{Gray}
Most-Popular &	0.2964	&	0.908403	&	0.973461	&	0	&	0.015807	&	0.043607	&	0	&	0.122245	&	0.149551	&	0	&	0.083855	&	0.116645	&	0	&	0.085254	&	0.114296	&	\\
\rowcolor{Gray}
ALS-WR & 	0.548427	&	0.808723	&	0.728263	&	0.000192	&	0.037148	&	0.011774	&	0.000926	&	0.092468	&	0.037826	&	0.000382	&	0.108875	&	0.03708	&	0	&	0.095247	&	0.026232	&	\\
\rowcolor{Gray}
IBCF	0.581572	&	0.783782	&	0.647155	&	0.000183	&	0.008705	&	0.002948	&	0.000926	&	0.021223	&	0.01396	&	0.001526	&	0.018261	&	0.015304	&	0	&	0.076197	&	0.013116	&	\\




\bottomrule
\end{tabular}
}
\caption{Cold-start item results - time splits}
\end{table}

\begin{table}[H]
\centering
\resizebox{\columnwidth}{!} & \multicolumn{1}{c}{40\%} & \multicolumn{1}{c}{75\%} & \multicolumn{1}{c}{10\%} & \multicolumn{1}{c}{40\%} & \multicolumn{1}{c}{75\%} & \multicolumn{1}{c}{10\%} &
\multicolumn{1}{c}{40\%} & \multicolumn{1}{c}{75\%} & \multicolumn{1}{c}{10\%} & \multicolumn{1}{c}{40\%} & \multicolumn{1}{c}{75\%} & \multicolumn{1}{c}{10\%} & \multicolumn{1}{c}{40\%} &
\multicolumn{1}{c}{75\%} & 	\\ \midrule 


%Pre Filterbots
\rowcolor{Gray}
Most-Popular &	0.653873	&	0.658753	&	0.657664	&	0.017394	&	0.005375	&	0.010465	&	0.023064	&	0.018463	&	0.01834	&	0.020592	&	0.021551	&	0.021091	&	0.023878	&	0.04188	&	0.041581	&	\\
\rowcolor{Gray}
ALS-WR &	0.669121	&	0.71492	&	0.747457	&	0.021574	&	0.012256	&	0.015096	&	0.015861	&	0.024894	&	0.017286	&	0.018796	&	0.030244	&	0.024057	&	0.035962	&	0.024274	&	0.039299	&	\\
\rowcolor{Gray}
IBCF &  0.525182	&	0.569436	&	0.618759	&	0.013867	&	0.003454	&	0.005097	&	0.00811	&	0.001534	&	0.010209	&	0.013058	&	0.009157	&	0.011673	&	0.026996	&	0.017607	&	0.018497	&	\\



\bottomrule
\end{tabular}
}
\caption{Cold-start user results -  time-split}
\end{table}

\begin{table}[H]
\centering
\resizebox{\columnwidth}{!} & \multicolumn{1}{c}{60\%} & \multicolumn{1}{c}{80\%} & \multicolumn{1}{c}{40\%} & \multicolumn{1}{c}{60\%} & \multicolumn{1}{c}{80\%} & \multicolumn{1}{c}{40\%} &
\multicolumn{1}{c}{60\%} & \multicolumn{1}{c}{80\%} & \multicolumn{1}{c}{40\%} & \multicolumn{1}{c}{60\%} & \multicolumn{1}{c}{80\%} & \multicolumn{1}{c}{40\%} & \multicolumn{1}{c}{60\%} &
\multicolumn{1}{c}{80\%} & 	\\ \midrule 


%Pre Filterbots
\rowcolor{Gray}
Most-Popular & 0.619492	&	0.593922	&	0.601734	&	0.005504	&	0.004694	&	0.003103	&	0.009514	&	0.007218	&	0.006234	&	0.006499	&	0.006066	&	0.004766	&	0.027778	&	0	&	0	&	\\	
\rowcolor{Gray}
ALS-WR & 0.600738	&	0.568137	&	0.570946	&	0.006208	&	0.001761	&	0.002545	&	0.004265	&	0.002625	&	0.003937	&	0.004766	&	0.005199	&	0.003466	&	0	&	0.027778	&	0.027778	&	\\
\rowcolor{Gray}
IBCF & 0.337922	&	0.340169	&	0.344166	&	0.000842	&	0.00055	&	0	&	0.002297	&	0	&	0	&	0.002166	&	0.0013	&	0	&	0.027778	&	0	&	0	&	\\
	
\bottomrule
\end{tabular}
}
\caption{Cold-start system results - time split}
\end{table}



\subsection{Random splits}

The following results shows the baseline cold-start performance of our models before applying the filterbots.



\begin{table}[H]
\centering
\resizebox{\columnwidth}{!} & \multicolumn{1}{c}{40\%} & \multicolumn{1}{c}{75\%} & \multicolumn{1}{c}{10\%} & \multicolumn{1}{c}{40\%} & \multicolumn{1}{c}{75\%} & \multicolumn{1}{c}{10\%} &
\multicolumn{1}{c}{40\%} & \multicolumn{1}{c}{75\%} & \multicolumn{1}{c}{10\%} & \multicolumn{1}{c}{40\%} & \multicolumn{1}{c}{75\%} & \multicolumn{1}{c}{10\%} & \multicolumn{1}{c}{40\%} &
\multicolumn{1}{c}{75\%} & 	\\ \midrule 


\rowcolor{Gray}
Most-Popular		& 0.3096225 & 0.90448 & 0.8469825 & 0 & 0.0267585 & 0.0234215 & 0 & 0.1207895 & 0.0943945 & 0 & 0.093357 & 0.07894715 & 0 & 0.08314335 & 	0.07447935 & \\
\rowcolor{Gray}
IBCF				& 0.585499 & 0.660409 & 0.7477435 & 0.000022 & 0.00571 & 0.005563 & 0.000308 & 0.019538 & 0.013896 & 0.000188 & 0.022364 & 0.0170475 & 0 & 0.034109 & 0.0284045 & \\ 
\rowcolor{Gray}
ALSWR				& 0.572094 & 0.8063845 &  0.698199 & 0.000124 & 0.024819 & 0.024362 & 0.00069 & 0.0745415 & 0.06881 & 0.000192 & 0.0819305 & 0.088515 & 	0.0025135 & 	0.071247 & 	0.071688 & \\


\bottomrule
\end{tabular}
}
\caption{Cold-start item results - random splits}
\end{table}

\begin{table}[H]
\centering
\resizebox{\columnwidth}{!} & \multicolumn{1}{c}{40\%} & \multicolumn{1}{c}{75\%} & \multicolumn{1}{c}{10\%} & \multicolumn{1}{c}{40\%} & \multicolumn{1}{c}{75\%} & \multicolumn{1}{c}{10\%} &
\multicolumn{1}{c}{40\%} & \multicolumn{1}{c}{75\%} & \multicolumn{1}{c}{10\%} & \multicolumn{1}{c}{40\%} & \multicolumn{1}{c}{75\%} & \multicolumn{1}{c}{10\%} & \multicolumn{1}{c}{40\%} &
\multicolumn{1}{c}{75\%} & 	\\ \midrule 


%Pre Filterbots
\rowcolor{Gray}
Most-Popular		& 0.7206055	&	0.7260445&	0.729684&	0.029688&	0.017628	& 0.0062015 &	0.039089 &	0.03653 &	0.021841 &	0.02732585 &	0.02744665 &	0.01734265 &	0.02675 & 0.02343615 &	0.01177485& \\
\rowcolor{Gray}
IBCF				& 0.538305	&	0.6320575 &	0.705002 &	0.021137 &	0.006926 &	0.0006575 &	0.0163445 &	0.0133165 &	0.0022905 &	0.020079 &	0.012804&	0.0042755 &	0.0310315&	0.033987&	0.012295& \\
\rowcolor{Gray}
ALSWR				& 0.7571865 &	0.741754&	0.796946&	0.0325415&	0.051159&	0.0305585&	0.0316075&	0.0401135&	0.0409645&	0.047797&	0.054792&	0.057723&	0.0428965&	0.063628& 	0.064841& \\


\bottomrule
\end{tabular}
}
\caption{Cold-start user results - random splits}
\end{table}

\begin{table}[H]
\centering
\resizebox{\columnwidth}{!} & \multicolumn{1}{c}{60\%} & \multicolumn{1}{c}{80\%} & \multicolumn{1}{c}{40\%} & \multicolumn{1}{c}{60\%} & \multicolumn{1}{c}{80\%} & \multicolumn{1}{c}{40\%} &
\multicolumn{1}{c}{60\%} & \multicolumn{1}{c}{80\%} & \multicolumn{1}{c}{40\%} & \multicolumn{1}{c}{60\%} & \multicolumn{1}{c}{80\%} & \multicolumn{1}{c}{40\%} & \multicolumn{1}{c}{60\%} &
\multicolumn{1}{c}{80\%} & 	\\ \midrule 


%Pre Filterbots
\rowcolor{Gray}
Most-Popular &	0.708515  &	0.6752165 &	0.663937  &	0.014044 &	0.012578 &	0.0139605 &	0.041488 & 0.0370545  & 0.033597	&	0.03214465 & 0.0297455 & 0.029427	&	0.02887465	 & 0.0243725 & 0.021974 & \\
\rowcolor{Gray}
IBCF 		 &	0.5730765 &	0.5847735 &	0.613753  &	0.003057 &	0.0034495 & 0.002457 &	0.009752 &	0.0125695 & 0.0086305   & 	0.021379   & 0.0136815 & 0.011022 	&	0.018308	 & 0.037721  & 	0.0271465  & \\
\rowcolor{Gray}
ALSWR		 &	0.7473595 &	0.646568  &	0.6094695 &	0.030684 &	0.0110905 &	0.0081595 &	0.076963 &	0.02659	& 	0.0199265	&	0.1011035 & 0.03916	& 0.028979	&	0.0823435 & 0.034093 &	0.018231  & \\	

	
\bottomrule
\end{tabular}
}
\caption{Cold-start system results - random splits}
\end{table}











%TODO - Average Table


The reason why most popular does so well on the cold-start item problem is due to the fact that we select the most popular items as test items.


\subsection{Filterbot Results}


\begin{table}\centering\resizebox{\columnwidth}{!}{\begin{tabular}{*{19}l}\toprule
& AUC &	MAP@20 &	T\_c &	T\_w &	T\_p &	P\_c &	P\_w &	P\_p &	R\_c &	R\_w &	R\_p &	MAP@20-click &	MAP@20-want &	MAP@20-purchase &	 \\
\midrule
% cold mahout-itembased
%mahout-itembased  0 1 1 1 1
item-10		&	0.533391 &	0.00143 &	3284 &	2652 &	248 &	25 &	13 &	1 &	0.007613 &	0.004902 &	0.004032 &	0.001675 &	0.000933 &	0.000168 &	 \\
item-40		&	0.686368 &	0.003133 &	2177 &	1787 &	163 &	35 &	22 &	3 &	0.016077 &	0.012311 &	0.018405 &	0.003349 &	0.002455 &	0.002319 &	 \\
item-75		&	0.615815 &	0.001894 &	940 &	762 &	75  &	6  &	4  &	0 &	0.006383 &	0.005249 &	0 &	0.002485 &	0.000421 &	0 &	 \\
system-40	&	0.662997 &	0.001756 &	2667 &	2446 &	284 &	19 &	15 &	8 &	0.007124 &	0.006132 &	0.028169 &	0.001457 &	0.000947 &	0.004393 &	 \\
system-80	&	0.607525 &	0.003504 &	2667 &	2446 &	284 &	35 &	23 &	5 &	0.013123 &	0.009403 &	0.017606 &	0.004988 &	0.002082 &	0.010538 &	 \\
system-60	&	0.630826 &	0.003032 &	2667 &	2446 &	284 &	36 &	30 &	7 &	0.013498 &	0.012265 &	0.024648 &	0.002693 &	0.002442 &	0.003903 &	 \\
user-10		&	0.644473 &	0.007364 &	3681 &	4869 &	414 &	22 &	40 &	13&	0.005977 &	0.008215 &	0.031401 &	0.001647 &	0.004722 &	0.005971 &	 \\
user-40		&	0.640768 &	0.003025 &	2469 &	3242 &	282 &	15 &	21 &	3 &	0.006075 &	0.006477 &	0.010638 &	0.003106 &	0.00299 &	0.001377 &	 \\
user-75		&	0.618034 &	0.000889 &	1059 &	1360 &	111 &	2  &	5  &	2 &	0.001889 &	0.003676 &	0.018018 &	0.000397 &	0.001004 &	0.00095 &	 \\

%mahout-itembased  0 0 1 0 1
item-10		&	0.541085 &	0.000543 &	3285 &	2664 &	246 &	6 &	7 &	0 &	0.001826 &	0.002628 &	0 &	0.000559 &	0.001333 &	0 &	 \\
item-40		&	0.762218 &	0.002345 &	2172 &	1790 &	171 &	18 &	13 &	3 &	0.008287 &	0.007263 &	0.017544 &	0.002429 &	0.001461 &	0.001182 &	 \\
item-75		&	0.707801 &	0.001898 &	926 &	773 &	79 &	7 &	1 &	1 &	0.007559 &	0.001294 &	0.012658 &	0.002506 &	0.00028 &	0.001812 &	 \\
system-40	&	0.624521 &	0.003668 &	2616 &	2506 &	275 &	39 &	23 &	2 &	0.014908 &	0.009178 &	0.007273 &	0.004677 &	0.003402 &	0.002049 &	 \\
system-60	&	0.570314 &	0.002879 &	2616 &	2506 &	275 &	36 &	49 &	3 &	0.013761 &	0.019553 &	0.010909 &	0.00188 &	0.005265 &	0.001684 &	 \\
system-80	&	0.55964  &	0.003104 &	2616 &	2506 &	275 &	25 &	52 &	9 &	0.009557 &	0.02075 &	0.032727 &	0.001973 &	0.004593 &	0.013081 &	 \\
user-10		&	0.618597 &	0.012549 &	3339 &	4847 &	439 &	49 &	56 &	15 &	0.014675 &	0.011554 &	0.034169 &	0.004038 &	0.009057 &	0.007973 &	 \\
user-40		&	0.70795  &	0.002649 &	2241 &	3241 &	287 &	9 &	22 &	6 &	0.004016 &	0.006788 &	0.020906 &	0.000514 &	0.002242 &	0.002546 &	 \\
user-75		&	0.711321 &	0.000948 &	967 &	1350 &	123 &	2 &	2 &	2 &	0.002068 &	0.001481 &	0.01626 &	0.000896 &	0.001133 &	0.000804 &	 \\

%mahout-svd 0 0 1 0 1
item-10		&	0.566195 &	0.000035 &	3318 &	2653 &	255 &	1   &	0   &	0  &	0.000301 &	0 		 &	0 		 &	0.000042 &	0 		 &	0 	     &	 \\
item-40		&	0.812083 &	0.030628 &	2215 &	1774 &	167 &	166 &	157 &	16 &	0.074944 &	0.088501 &	0.095808 &	0.026215 &	0.03208  &	0.035591 &	 \\
item-75		&	0.795006 &	0.045217 &	968  &	758  &	63  &	103 &	119 &	11 &	0.106405 &	0.156992 &	0.174603 &	0.033899 &	0.055343 &	0.057762 &	 \\
system-40	&	0.681467 &	0.018522 &	2624 &	2502 &	271 &	114 &	123 &	19 &	0.043445 &	0.049161 &	0.070111 &	0.014967 &	0.022078 &	0.018751 &	 \\
system-60	&	0.620744 &	0.010318 &	2624 &	2502 &	271 &	51  &	90  &	9  &	0.019436 &	0.035971 &	0.03321  &	0.00906  &	0.01655  &	0.013523 &	 \\
system-80	&	0.587088 &	0.008626 &	2624 &	2502 &	271 &	58  &	63  &	4  &	0.022104 &	0.02518  &	0.01476  &	0.010438 &	0.005166 &	0.006268 &	 \\
user-10		&	0.695013 &	0.045986 &	3611 &	5039 &	319 &	114 &	216 &	13 &	0.03157  &	0.042866 &	0.040752 &	0.015062 &	0.039264 &	0.016756 &	 \\
user-40		&	0.767194 &	0.045413 &	2451 &	3342 &	200 &	128 &	180 &	11 &	0.052224 &	0.05386  &	0.055 	 &	0.024119 &	0.037459 &	0.008932 &	 \\
user-75		&	0.788278 &	0.019044 &	1029 &	1420 &	86  &	32  &	65  &	3  &	0.031098 &	0.045775 &	0.034884 &	0.008769 &	0.03177  &	0.010338 &	 \\

%mahout-svd 1 1 1 1 1
item-10		&	0.637067 &	0.000114 &	3252 &	2625 &	254 &	2 &	1 &	1 &	0.000615 &	0.000381 &	0.003937 &	0.00012 &	0.000015 &	0.000441 &	 \\
item-40		&	0.79768 &	0.026056 &	2127 &	1795 &	168 &	149 &	132 &	17 &	0.070052 &	0.073538 &	0.10119 &	0.02535 &	0.026429 &	0.021847 &	 \\
item-75		&	0.762618 &	0.038596 &	939 &	748 &	72 &	108 &	90 &	10 &	0.115016 &	0.120321 &	0.138889 &	0.034542 &	0.045579 &	0.066274 &	 \\
system-40	&	0.726329 &	0.022999 &	2626 &	2484 &	287 &	114 &	155 &	18 &	0.043412 &	0.062399 &	0.062718 &	0.015076 &	0.033586 &	0.035945 &	 \\
system-60	& 	0.677643 &	0.009503 &	2626 &	2484 &	287 &	66 &	96 &	7 &	0.025133 &	0.038647 &	0.02439 &	0.00645 &	0.012291 &	0.014648 &	 \\
system-80	&	0.666988 &	0.011568 &	2626 &	2484 &	287 &	61 &	68 &	9 &	0.023229 &	0.027375 &	0.031359 &	0.012111 &	0.015055 &	0.018528 &	 \\
user-10		&	0.696383 &	0.03936 &	3631 &	4954 &	365 &	112 &	181 &	14 &	0.030845 &	0.036536 &	0.038356 &	0.023258 &	0.028335 &	0.021611 &	 \\
user-40		&	0.780863 &	0.045689 &	2448 &	3302 &	237 &	115 &	190 &	17 &	0.046977 &	0.057541 &	0.07173 &	0.022413 &	0.036282 &	0.018864 &	 \\
user-75		&	0.813625 &	0.0152 &	1029 &	1389 &	113 &	38 &	58 &	5 &	0.036929 &	0.041757 &	0.044248 &	0.012223 &	0.010903 &	0.025801 &	 \\



%mahout-itembased 1 1 1 1 1
item-10		&	0.494814 &	0.000643 &	3260 &	2634 &	242 &	12 &	16 &	0 &	0.003681 &	0.006074 &	0 &	0.000606 &	0.000914 &	0 &	 \\
item-40		&	0.541715 &	0.001378 &	2155 &	1777 &	163 &	15 &	10 &	0 &	0.006961 &	0.005627 &	0 &	0.000992 &	0.001786 &	0 &	 \\
item-75		&	0.500892 &	0.000324 &	938 &	751 &	74 &	3  &	1 &	0 &	0.003198 &	0.001332 &	0 &	0.000666 &	0.000047 &	0 &	 \\
system-40	&	0.603068 &	0.000799 &	2618 &	2529 &	250 &	9  &	12 &	1 &	0.003438 &	0.004745 &	0.004 &	0.000424 &	0.001296 &	0.000041 &	 \\
system-60	&	0.613217 &	0.004139 &	2618 &	2529 &	250 &	12 &	26 &	5 &	0.004584 &	0.010281 &	0.02 &	0.002503 &	0.002674 &	0.010649 &	 \\
system-80	&	0.612799 &	0.001529 &	2618 &	2529 &	250 &	16 &	26 &	3 &	0.006112 &	0.010281 &	0.012 &	0.001153 &	0.002332 &	0.002503 &	 \\
user-10		&	0.573466 &	0.008001 &	3775 &	4149 &	536 &	20 &	35 &	9 &	0.005298 &	0.008436 &	0.016791 &	0.001391 &	0.006385 &	0.003986 &	 \\
user-40		&	0.522617 &	0.003985 &	2557 &	2718 &	383 &	14 &	14 &	7 &	0.005475 &	0.005151 &	0.018277 &	0.001056 &	0.002056 &	0.003096 &	 \\
user-75		&	0.503525 &	0.000301 &	1094 &	1155 &	139 &	3 &	3 &	2 &	0.002742 &	0.002597 &	0.014388 &	0.000583 &	0.000258 &	0.00044 &	 \\



%mahout-itembased 0 0 0 0 1
item-10 &   0.625537 &  0.000211 &  3237 &  2668 &  250 &   13 &    8 & 0 & 0.004016 &  0.002999 &  0 & 0.000209 &  0.000339 &  0 &  \\
item-40 &   0.691961 &  0.007045 &  2194 &  1749 &  164 &   20 &    7 & 2 & 0.009116 &  0.004002 &  0.012195 &  0.006667 &  0.003599 &  0.011299 &   \\
item-75 &   0.659555 &  0.004775 &  894 &   795 &   80 &    8 & 1 & 1 & 0.008949 &  0.001258 &  0.0125 &    0.005466 &  0.000326 &  0.007143 &   \\
user-10 &   0.643336 &  0.015582 &  3859 &  4018 &  549 &   42 &    83 &    14 &    0.010884 &  0.020657 &  0.025501 &  0.013403 &  0.013319 &  0.003465 &   \\
user-40 &   0.68084 &   0.002554 &  2586 &  2683 &  365 &   8 & 10 &    3 & 0.003094 &  0.003727 &  0.008219 &  0.001092 &  0.001616 &  0.002179 &   \\
user-75 &   0.661735 &  0.000117 &  1106 &  1127 &  146 &   1 & 1 & 0 & 0.000904 &  0.000887 &  0 & 0.000092 &  0.000101 &  0 &  \\
system-40   &   0.652882 &  0.004676 &  2632 &  2480 &  284 &   30 &    27 &    6 & 0.011398 &  0.010887 &  0.021127 &  0.004748 &  0.006179 &  0.010335 &   \\
system-60   &   0.620468 &  0.005518 &  2632 &  2480 &  284 &   32 &    53 &    7 & 0.012158 &  0.021371 &  0.024648 &  0.002817 &  0.009307 &  0.009932 &   \\
system-80   &   0.617179 &  0.006085 &  2632 &  2480 &  284 &   30 &    45 &    7 & 0.011398 &  0.018145 &  0.024648 &  0.003303 &  0.008998 &  0.003992 &   \\

%mahout-svd 1 1 0 0 0
item-10 &   0.583085 &  0.000077 &  3218 &  2635 &  248 &   2 & 0 & 1 & 0.000622 &  0 & 0.004032 &  0.000076 &  0 & 0.000156 &   \\
item-40 &   0.741617 &  0.018007 &  2138 &  1769 &  163 &   120 &   105 &   11 &    0.056127 &  0.059356 &  0.067485 &  0.016841 &  0.014988 &  0.033753 &   \\
item-75 &   0.771983 &  0.028084 &  928 &   760 &   67 &    83 &    40 &    8 & 0.08944 &   0.052632 &  0.119403 &  0.027894 &  0.024632 &  0.018937 &   \\
system-40   &   0.627412 &  0.009427 &  2580 &  2517 &  299 &   71 &    79 &    9 & 0.027519 &  0.031387 &  0.0301 &    0.008094 &  0.009918 &  0.011383 &   \\
system-60   &   0.570877 &  0.006754 &  2580 &  2517 &  299 &   37 &    45 &    7 & 0.014341 &  0.017878 &  0.023411 &  0.006498 &  0.006158 &  0.00809 &    \\
system-80   &   0.567509 &  0.006806 &  2580 &  2517 &  299 &   34 &    42 &    6 & 0.013178 &  0.016687 &  0.020067 &  0.007234 &  0.004787 &  0.003634 &   \\
user-10 &   0.654865 &  0.031627 &  3301 &  4964 &  317 &   98 &    155 &   6 & 0.029688 &  0.031225 &  0.018927 &  0.011458 &  0.025798 &  0.003296 &   \\
user-40 &   0.733925 &  0.029318 &  2205 &  3327 &  206 &   84 &    153 &   6 & 0.038095 &  0.045987 &  0.029126 &  0.010782 &  0.028561 &  0.008397 &   \\
user-75 &   0.772863 &  0.009513 &  927 &   1408 &  86 &    23 &    40 &    3 & 0.024811 &  0.028409 &  0.034884 &  0.004768 &  0.011167 &  0.007794 &   \\

%mahout-itembased 1 1 0 0 0
item-10 &   0.489759 &  0.000077 &  3218 &  2635 &  248 &   3 & 0 & 0 & 0.000932 &  0 & 0 & 0.0001 &    0 & 0 &  \\
item-40 &   0.52456 &   0.007423 &  2138 &  1769 &  163 &   44 &    31 &    2 & 0.02058 &   0.017524 &  0.01227 &   0.008303 &  0.00558 &   0.003672 &   \\
item-75 &   0.475673 &  0.004157 &  928 &   760 &   67 &    12 &    9 & 1 & 0.012931 &  0.011842 &  0.014925 &  0.004607 &  0.002461 &  0.002232 &   \\
system-40   &   0.553991 &  0.004181 &  2580 &  2517 &  299 &   48 &    18 &    5 & 0.018605 &  0.007151 &  0.016722 &  0.005247 &  0.002532 &  0.005386 &   \\
system-60   &   0.567005 &  0.00248 &   2580 &  2517 &  299 &   26 &    25 &    2 & 0.010078 &  0.009932 &  0.006689 &  0.002097 &  0.002846 &  0.00045 &    \\
system-80   &   0.565274 &  0.003118 &  2580 &  2517 &  299 &   34 &    28 &    5 & 0.013178 &  0.011124 &  0.016722 &  0.002677 &  0.004938 &  0.006441 &   \\
user-10 &   0.503302 &  0.010576 &  3301 &  4964 &  317 &   43 &    60 &    2 & 0.013026 &  0.012087 &  0.006309 &  0.006465 &  0.006554 &  0.000448 &   \\
user-40 &   0.494097 &  0.003896 &  2205 &  3327 &  206 &   16 &    19 &    2 & 0.007256 &  0.005711 &  0.009709 &  0.002782 &  0.002324 &  0.000271 &   \\
user-75 &   0.459582 &  0.000936 &  927 &   1408 &  86 &    5 & 2 & 2 & 0.005394 &  0.00142 &   0.023256 &  0.001038 &  0.000258 &  0.000888 &   \\

%mahout-svd 0 0 0 0 1 
user-10 &   0.706584 &  0.034523 &  3859 &  4018 &  549 &   115 &   148 &   17 &    0.0298 &    0.036834 &  0.030965 &  0.023269 &  0.03081 &   0.007341 &   \\
user-40 &   0.762294 &  0.022352 &  2586 &  2683 &  365 &   77 &    128 &   11 &    0.029776 &  0.047708 &  0.030137 &  0.012166 &  0.026376 &  0.007249 &   \\
user-75 &   0.805556 &  0.014329 &  1106 &  1127 &  146 &   42 &    41 &    8 & 0.037975 &  0.03638 &   0.054795 &  0.00623 &   0.020685 &  0.007444 &   \\
system-40   &   0.68831 &   0.014679 &  2632 &  2480 &  284 &   93 &    97 &    11 &    0.035334 &  0.039113 &  0.038732 &  0.014615 &  0.010681 &  0.011093 &   \\
system-60   &   0.634666 &  0.009741 &  2632 &  2480 &  284 &   60 &    50 &    7 & 0.022796 &  0.020161 &  0.024648 &  0.008924 &  0.00915 &   0.005175 &   \\
system-80   &   0.615021 &  0.006068 &  2632 &  2480 &  284 &   44 &    52 &    6 & 0.016717 &  0.020968 &  0.021127 &  0.003233 &  0.011019 &  0.007315 &   \\
item-10 &   0.615937 &  0.00005 &   3237 &  2668 &  250 &   3 & 3 & 0 & 0.000927 &  0.001124 &  0 & 0.000073 &  0.000053 &  0 &  \\
item-40 &   0.776737 &  0.029282 &  2194 &  1749 &  164 &   155 &   116 &   16 &    0.070647 &  0.066324 &  0.097561 &  0.025718 &  0.033229 &  0.023693 &   \\
item-75 &   0.781276 &  0.030254 &  894 &   795 &   80 &    90 &    90 &    10 &    0.100671 &  0.113208 &  0.125 & 0.030844 &  0.026848 &  0.024821 &   \\

\bottomrule\end{tabular}}\caption{filterbots]}\end{table}

%TODO - Test more filterbot combinations

%TODO - 

%BASELINES TIME 0 0 0 0 0 
\begin{table}\centering\resizebox{\columnwidth}{!}{\begin{tabular}{*{19}l}\toprule
 & AUC &	MAP@20 &	T\_c &	T\_w &	T\_p &	P\_c &	P\_w &	P\_p &	R\_c &	R\_w &	R\_p &	MAP@20-click &	MAP@20-want &	MAP@20-purchase &	 \\
\midrule
item-10 &	0.2964 &	0 &	3241 &	2620.5 &	243 &	0 &	0 &	0 &	0 &	0 &	0 &	0 &	0 &	0 &	 \\
item-75 &	0.973461 &	0.043607 &	989.5 &	712 &	52.5 &	148 &	83 &	6 &	0.149551 &	0.116645 &	0.114296 &	0.050583 &	0.029793 &	0.064658 &	 \\
system-40 &	0.619492 &	0.005504 &	3048 &	2308 &	36 &	29 &	15 &	1 &	0.009514 &	0.006499 &	0.027778 &	0.005144 &	0.003966 &	0.005208 &	 \\
system-80 &	0.601734 &	0.003103 &	3048 &	2308 &	36 &	19 &	11 &	0 &	0.006234 &	0.004766 &	0 &	0.002911 &	0.002184 &	0 &	 \\
system-60 &	0.593922 &	0.004694 &	3048 &	2308 &	36 &	22 &	14 &	0 &	0.007218 &	0.006066 &	0 &	0.004547 &	0.003213 &	0 &	 \\
item-40 &	0.908403 &	0.015807 &	2221 &	1699.5 &	152.5 &	271.5 &	142.5 &	13 &	0.122245 &	0.083855 &	0.085254 &	0.018698 &	0.01177 &	0.013565 &	 \\
user-75 &	0.657664 &	0.005375 &	1000.5 &	1216 &	96 &	18 &	27 &	3.5 &	0.018463 &	0.021551 &	0.04188 &	0.00594 &	0.004096 &	0.00596 &	 \\
user-10 &	0.653873 &	0.017394 &	3673 &	4180.5 &	336.5 &	82 &	90 &	8 &	0.023064 &	0.020592 &	0.023878 &	0.009412 &	0.009061 &	0.007878 &	 \\
user-40 &	0.658753 &	0.010465 &	2416 &	2844.5 &	216.5 &	43 &	62.5 &	9 &	0.01834 &	0.021091 &	0.041581 &	0.006525 &	0.006448 &	0.009125 &	 \\
avgs	 &	0.662634 &	0.011772 &	2520.555556 &	2244.111111 &	133.888889 &	70.277778 &	49.444444 &	4.5 &	0.039403 &	0.03123 &	0.037185 &	0.011529 &	0.007837 &	0.011822 &	\\
\bottomrule\end{tabular}}\caption{cold item recommender-MostPopular}\end{table}
\begin{table}\centering\resizebox{\columnwidth}{!}{\begin{tabular}{*{19}l}\toprule
 & AUC &	MAP@20 &	T\_c &	T\_w &	T\_p &	P\_c &	P\_w &	P\_p &	R\_c &	R\_w &	R\_p &	MAP@20-click &	MAP@20-want &	MAP@20-purchase &	 \\
\midrule
item-10 &	0.548427 &	0.000192 &	3241 &	2620.5 &	243 &	3 &	1 &	0 &	0.000926 &	0.000382 &	0 &	0.000216 &	0.000047 &	0 &	 \\
item-40 &	0.728263 &	0.011774 &	2221 &	1699.5 &	152.5 &	84 &	63 &	4 &	0.037826 &	0.03708 &	0.026232 &	0.010776 &	0.014745 &	0.005461 &	 \\
system-40 &	0.600738 &	0.006208 &	3048 &	2308 &	36 &	13 &	11 &	0 &	0.004265 &	0.004766 &	0 &	0.005055 &	0.014151 &	0 &	 \\
system-80 &	0.570946 &	0.002545 &	3048 &	2308 &	36 &	12 &	8 &	1 &	0.003937 &	0.003466 &	0.027778 &	0.001969 &	0.002149 &	0.041667 &	 \\
system-60 &	0.568137 &	0.001761 &	3048 &	2308 &	36 &	8 &	12 &	1 &	0.002625 &	0.005199 &	0.027778 &	0.00108 &	0.008555 &	0.004167 &	 \\
item-75 &	0.808723 &	0.037148 &	989.5 &	712 &	52.5 &	91.5 &	77.5 &	5 &	0.092468 &	0.108875 &	0.095247 &	0.031812 &	0.040076 &	0.019101 &	 \\
user-75 &	0.747457 &	0.012256 &	1000.5 &	1216 &	96 &	24.5 &	38 &	2 &	0.024894 &	0.030244 &	0.024274 &	0.009066 &	0.016027 &	0.02484 &	 \\
user-10 &	0.669121 &	0.021574 &	3673 &	4180.5 &	336.5 &	56.5 &	82 &	12 &	0.015861 &	0.018796 &	0.035962 &	0.009461 &	0.013781 &	0.020946 &	 \\
user-40 &	0.71492 &	0.015096 &	2416 &	2844.5 &	216.5 &	40.5 &	71 &	8.5 &	0.017286 &	0.024057 &	0.039299 &	0.00768 &	0.012863 &	0.015004 &	 \\
avgs	 &	0.661859 &	0.012061 &	2520.555556 &	2244.111111 &	133.888889 &	37 &	40.388889 &	3.722222 &	0.022232 &	0.025874 &	0.03073 &	0.008568 &	0.013599 &	0.014576 &	\\
\bottomrule\end{tabular}}\caption{cold mahout-svd}\end{table}
\begin{table}\centering\resizebox{\columnwidth}{!}{\begin{tabular}{*{19}l}\toprule
 & AUC &	MAP@20 &	T\_c &	T\_w &	T\_p &	P\_c &	P\_w &	P\_p &	R\_c &	R\_w &	R\_p &	MAP@20-click &	MAP@20-want &	MAP@20-purchase &	 \\
\midrule
item-10 &	0.581572 &	0.000183 &	3241 &	2620.5 &	243 &	3 &	4 &	0 &	0.000926 &	0.001526 &	0 &	0.000168 &	0.000166 &	0 &	 \\
item-40 &	0.647155 &	0.002948 &	2221 &	1699.5 &	152.5 &	31 &	26 &	2 &	0.01396 &	0.015304 &	0.013116 &	0.002202 &	0.004557 &	0.000428 &	 \\
system-40 &	0.337922 &	0.000842 &	3048 &	2308 &	36 &	7 &	5 &	1 &	0.002297 &	0.002166 &	0.027778 &	0.000625 &	0.001511 &	0.00463 &	 \\
system-80 &	0.344166 &	0 &	3048 &	2308 &	36 &	0 &	0 &	0 &	0 &	0 &	0 &	0 &	0 &	0 &	 \\
system-60 &	0.340169 &	0.00055 &	3048 &	2308 &	36 &	0 &	3 &	0 &	0 &	0.0013 &	0 &	0 &	0.003669 &	0 &	 \\
item-75 &	0.783782 &	0.008705 &	989.5 &	712 &	52.5 &	21 &	13 &	4 &	0.021223 &	0.018261 &	0.076197 &	0.008209 &	0.00558 &	0.026017 &	 \\
user-75 &	0.618759 &	0.003454 &	1000.5 &	1216 &	96 &	1.5 &	11.5 &	1.5 &	0.001534 &	0.009157 &	0.017607 &	0.000947 &	0.004815 &	0.004309 &	 \\
user-10 &	0.525182 &	0.013867 &	3673 &	4180.5 &	336.5 &	29 &	56.5 &	9 &	0.00811 &	0.013058 &	0.026996 &	0.004283 &	0.009368 &	0.012624 &	 \\
user-40 &	0.569436 &	0.005097 &	2416 &	2844.5 &	216.5 &	24 &	34.5 &	4 &	0.010209 &	0.011673 &	0.018497 &	0.003462 &	0.004267 &	0.003834 &	 \\
avgs	 &	0.527572 &	0.003961 &	2520.555556 &	2244.111111 &	133.888889 &	12.944444 &	17.055556 &	2.388889 &	0.006473 &	0.008049 &	0.020021 &	0.002211 &	0.00377 &	0.00576 &	\\
\bottomrule\end{tabular}}\caption{cold mahout-itembased}\end{table}

%Timesplits 0,0,1,0,1
\begin{table}\centering\resizebox{\columnwidth}{!}{\begin{tabular}{*{19}l}\toprule
 & AUC &	MAP@20 &	T\_c &	T\_w &	T\_p &	P\_c &	P\_w &	P\_p &	R\_c &	R\_w &	R\_p &	MAP@20-click &	MAP@20-want &	MAP@20-purchase &	 \\
\midrule
item-10 &	0.35572 &	0 &	3322.5 &	2617 &	248 &	0 &	0 &	0 &	0 &	0 &	0 &	0 &	0 &	0 &	 \\
item-75 &	0.975673 &	0.048628 &	1011.5 &	711.5 &	55 &	169 &	94.5 &	9 &	0.167041 &	0.132862 &	0.16369 &	0.053837 &	0.037729 &	0.059581 &	 \\
system-40 &	0.629744 &	0.003438 &	3048 &	2308 &	36 &	23 &	8 &	0 &	0.007546 &	0.003466 &	0 &	0.00309 &	0.004938 &	0 &	 \\
system-80 &	0.580986 &	0.002632 &	3048 &	2308 &	36 &	23 &	8 &	0 &	0.007546 &	0.003466 &	0 &	0.002513 &	0.001882 &	0 &	 \\
system-60 &	0.59533 &	0.003069 &	3048 &	2308 &	36 &	20 &	9 &	1 &	0.006562 &	0.003899 &	0.027778 &	0.00276 &	0.003121 &	0.008264 &	 \\
item-40 &	0.912929 &	0.035728 &	2278 &	1696.5 &	155 &	277.5 &	160 &	17 &	0.121801 &	0.094329 &	0.109696 &	0.041795 &	0.026738 &	0.03782 &	 \\
user-75 &	0.654875 &	0.006714 &	1094 &	1327.5 &	91 &	24.5 &	25.5 &	3 &	0.02216 &	0.018878 &	0.032076 &	0.009061 &	0.002829 &	0.015892 &	 \\
user-10 &	0.65313 &	0.018681 &	3952 &	4562.5 &	391 &	91 &	91 &	7.5 &	0.022691 &	0.019712 &	0.018695 &	0.010976 &	0.008706 &	0.010585 &	 \\
user-40 &	0.655761 &	0.012391 &	2609 &	3131 &	218 &	54 &	62.5 &	7.5 &	0.020402 &	0.019727 &	0.033626 &	0.007772 &	0.006141 &	0.015207 &	 \\
avgs	 &	0.668239 &	0.014587 &	2601.222222 &	2330 &	140.666667 &	75.777778 &	50.944444 &	5 &	0.04175 &	0.032927 &	0.04284 &	0.014645 &	0.010231 &	0.016372 &	\\
\bottomrule\end{tabular}}\caption{cold item recommender-MostPopular}\end{table}
\begin{table}\centering\resizebox{\columnwidth}{!}{\begin{tabular}{*{19}l}\toprule
 & AUC &	MAP@20 &	T\_c &	T\_w &	T\_p &	P\_c &	P\_w &	P\_p &	R\_c &	R\_w &	R\_p &	MAP@20-click &	MAP@20-want &	MAP@20-purchase &	 \\
\midrule
item-10 &	0.625073 &	0.000518 &	3322.5 &	2617 &	248 &	5 &	7.5 &	0 &	0.001505 &	0.002867 &	0 &	0.000437 &	0.000596 &	0 &	 \\
item-40 &	0.727069 &	0.013534 &	2278 &	1696.5 &	155 &	89.5 &	76.5 &	6 &	0.039285 &	0.045097 &	0.038716 &	0.014394 &	0.015436 &	0.009411 &	 \\
system-40 &	0.617729 &	0.003569 &	3048 &	2308 &	36 &	16 &	12 &	0 &	0.005249 &	0.005199 &	0 &	0.00294 &	0.003113 &	0 &	 \\
system-80 &	0.576757 &	0.001443 &	3048 &	2308 &	36 &	9 &	7 &	0 &	0.002953 &	0.003033 &	0 &	0.001139 &	0.001161 &	0 &	 \\
system-60 &	0.570766 &	0.002491 &	3048 &	2308 &	36 &	11 &	14 &	1 &	0.003609 &	0.006066 &	0.027778 &	0.001242 &	0.004649 &	0.005682 &	 \\
item-75 &	0.785433 &	0.036564 &	1011.5 &	711.5 &	55 &	98 &	80 &	6 &	0.096876 &	0.112471 &	0.109127 &	0.033223 &	0.035562 &	0.043011 &	 \\
user-75 &	0.737931 &	0.011515 &	1094 &	1327.5 &	91 &	25 &	34.5 &	3 &	0.022635 &	0.025434 &	0.030601 &	0.007306 &	0.011666 &	0.00666 &	 \\
user-10 &	0.672386 &	0.028034 &	3952 &	4562.5 &	391 &	65 &	83 &	17.5 &	0.016251 &	0.0179 &	0.041984 &	0.014275 &	0.016082 &	0.022662 &	 \\
user-40 &	0.707324 &	0.016305 &	2609 &	3131 &	218 &	39 &	62.5 &	9 &	0.014753 &	0.019615 &	0.03672 &	0.005155 &	0.017507 &	0.019089 &	 \\
avgs	 &	0.668941 &	0.012664 &	2601.222222 &	2330 &	140.666667 &	39.722222 &	41.888889 &	4.722222 &	0.022568 &	0.026409 &	0.031658 &	0.008901 &	0.011752 &	0.011835 &	\\
\bottomrule\end{tabular}}\caption{cold mahout-svd}\end{table}
\begin{table}\centering\resizebox{\columnwidth}{!}{\begin{tabular}{*{19}l}\toprule
 & AUC &	MAP@20 &	T\_c &	T\_w &	T\_p &	P\_c &	P\_w &	P\_p &	R\_c &	R\_w &	R\_p &	MAP@20-click &	MAP@20-want &	MAP@20-purchase &	 \\
\midrule
item-10 &	0.558101 &	0.000614 &	3322.5 &	2617 &	248 &	11 &	5 &	1 &	0.003311 &	0.001911 &	0.004033 &	0.000548 &	0.000845 &	0.000313 &	 \\
item-40 &	0.758747 &	0.002795 &	2278 &	1696.5 &	155 &	23.5 &	11 &	3 &	0.010315 &	0.006484 &	0.019358 &	0.001654 &	0.004507 &	0.00173 &	 \\
system-40 &	0.410766 &	0.000645 &	3048 &	2308 &	36 &	12 &	3 &	0 &	0.003937 &	0.0013 &	0 &	0.000897 &	0.000154 &	0 &	 \\
system-80 &	0.330608 &	0.000442 &	3048 &	2308 &	36 &	2 &	6 &	0 &	0.000656 &	0.0026 &	0 &	0.000151 &	0.000805 &	0 &	 \\
system-60 &	0.360732 &	0.000567 &	3048 &	2308 &	36 &	3 &	7 &	1 &	0.000984 &	0.003033 &	0.027778 &	0.000153 &	0.003944 &	0.004545 &	 \\
item-75 &	0.78616 &	0.00437 &	1011.5 &	711.5 &	55 &	15.5 &	6.5 &	0 &	0.015321 &	0.009145 &	0 &	0.005092 &	0.003701 &	0 &	 \\
user-75 &	0.66947 &	0.000629 &	1094 &	1327.5 &	91 &	1.5 &	2.5 &	0 &	0.001355 &	0.001868 &	0 &	0.000213 &	0.000751 &	0 &	 \\
user-10 &	0.587419 &	0.016058 &	3952 &	4562.5 &	391 &	26 &	58.5 &	13 &	0.006526 &	0.012636 &	0.030767 &	0.009858 &	0.010059 &	0.013228 &	 \\
user-40 &	0.659185 &	0.003011 &	2609 &	3131 &	218 &	7 &	19.5 &	1.5 &	0.00264 &	0.006135 &	0.005515 &	0.000622 &	0.003465 &	0.000174 &	 \\
avgs	 &	0.569021 &	0.003237 &	2601.222222 &	2330 &	140.666667 &	11.277778 &	13.222222 &	2.166667 &	0.005005 &	0.005012 &	0.009717 &	0.002132 &	0.003137 &	0.002221 &	\\
\bottomrule\end{tabular}}\caption{cold mahout-itembased}\end{table}



%SVD - TimeSplits 1 1 1 1 1
%item-10 &	0.502866 &	0.001702 &	3307 &	2614 &	242 &	22 &	14 &	1 &	0.006653 &	0.005356 &	0.004132 &	0.001409 &	0.001518 &	0.000781 &	 \\
%item-40 &	0.528849 &	0.001701 &	2271 &	1690 &	151 &	15 &	9 &	0 &	0.006605 &	0.005325 &	0 &	0.001652 &	0.001446 &	0 &	 \\
%system-40 &	0.42917 &	0.000196 &	3048 &	2308 &	36 &	2 &	2 &	0 &	0.000656 &	0.000867 &	0 &	0.000153 &	0.000124 &	0 &	 \\
%system-80 &	0.40721 &	0.000253 &	3048 &	2308 &	36 &	5 &	2 &	0 &	0.00164 &	0.000867 &	0 &	0.000213 &	0.00012 &	0 &	 \\
%system-60 &	0.415846 &	0.000999 &	3048 &	2308 &	36 &	5 &	1 &	0 &	0.00164 &	0.000433 &	0 &	0.001044 &	0.000219 &	0 &	 \\
%item-75 &	0.515971 &	0.001253 &	1010 &	708 &	53 &	9 &	6 &	1 &	0.008911 &	0.008475 &	0.018868 &	0.00092 &	0.001444 &	0.002976 &	 \\
%user-75 &	0.515962 &	0.001336 &	1062 &	1559 &	70 &	1 &	10 &	1 &	0.000942 &	0.006414 &	0.014286 &	0.000751 &	0.0016 &	0.002232 &	 \\
%user-10 &	0.580926 &	0.005989 &	3844 &	5281 &	430 &	15 &	47 &	11 &	0.003902 &	0.0089 &	0.025581 &	0.001671 &	0.004785 &	0.004144 &	 \\
%user-40 &	0.558447 &	0.002253 &	2536 &	3620 &	230 &	4 &	19 &	1 &	0.001577 &	0.005249 &	0.004348 &	0.000589 &	0.001866 &	0.001064 &	 \\
%avgs	 &	0.495028 &	0.001742 &	2574.888889 &	2488.444444 &	142.666667 &	8.666667 &	12.222222 &	1.666667 &	0.003614 &	0.004654 &	0.007468 &	0.000934 &	0.001458 &	0.001244 &	\\
%
%%ItemBased - Timesplits 1 1 1 1 1
%item-10 &	0.639232 &	0.000477 &	3307 &	2614 &	242 &	6 &	9 &	0 &	0.001814 &	0.003443 &	0 &	0.00041 &	0.000549 &	0 &	 \\
%item-40 &	0.726871 &	0.01346 &	2271 &	1690 &	151 &	96 &	73 &	6 &	0.042272 &	0.043195 &	0.039735 &	0.012657 &	0.019136 &	0.010179 &	 \\
%system-40 &	0.534059 &	0.003978 &	3048 &	2308 &	36 &	13 &	10 &	0 &	0.004265 &	0.004333 &	0 &	0.003523 &	0.001789 &	0 &	 \\
%system-80 &	0.542178 &	0.001593 &	3048 &	2308 &	36 &	12 &	7 &	0 &	0.003937 &	0.003033 &	0 &	0.001467 &	0.001532 &	0 &	 \\
%system-60 &	0.52987 &	0.004118 &	3048 &	2308 &	36 &	16 &	12 &	2 &	0.005249 &	0.005199 &	0.055556 &	0.003343 &	0.002371 &	0.005468 &	 \\
%item-75 &	0.765866 &	0.037104 &	1010 &	708 &	53 &	97 &	74 &	4 &	0.09604 &	0.10452 &	0.075472 &	0.034926 &	0.036877 &	0.010268 &	 \\
%user-75 &	0.718682 &	0.009978 &	1062 &	1559 &	70 &	22 &	63 &	4 &	0.020716 &	0.040411 &	0.057143 &	0.00273 &	0.013868 &	0.011675 &	 \\
%user-10 &	0.658746 &	0.02711 &	3844 &	5281 &	430 &	68 &	129 &	29 &	0.01769 &	0.024427 &	0.067442 &	0.007861 &	0.018109 &	0.038802 &	 \\
%user-40 &	0.689484 &	0.017514 &	2536 &	3620 &	230 &	55 &	97 &	14 &	0.021688 &	0.026796 &	0.06087 &	0.006168 &	0.014466 &	0.017258 &	 \\
%avgs	 &	0.644999 &	0.012815 &	2574.888889 &	2488.444444 &	142.666667 &	42.777778 &	52.666667 &	6.555556 &	0.023741 &	0.028373 &	0.03958 &	0.00812 &	0.012078 &	0.010406 &	\\

%TODO -

\begin{table}\centering\resizebox{\columnwidth}{!}{\begin{tabular}{*{19}l}\toprule
%SVD - TimeSplits  0 0 1 0 1
item-10 &	0.502866 &	0.001702 &	3307 &	2614 &	242 &	22 &	14 &	1 &	0.006653 &	0.005356 &	0.004132 &	0.001409 &	0.001518 &	0.000781 &	 \\
item-40 &	0.528849 &	0.001701 &	2271 &	1690 &	151 &	15 &	9 &	0 &	0.006605 &	0.005325 &	0 &	0.001652 &	0.001446 &	0 &	 \\
system-40 &	0.42917 &	0.000196 &	3048 &	2308 &	36 &	2 &	2 &	0 &	0.000656 &	0.000867 &	0 &	0.000153 &	0.000124 &	0 &	 \\
system-80 &	0.40721 &	0.000253 &	3048 &	2308 &	36 &	5 &	2 &	0 &	0.00164 &	0.000867 &	0 &	0.000213 &	0.00012 &	0 &	 \\
system-60 &	0.415846 &	0.000999 &	3048 &	2308 &	36 &	5 &	1 &	0 &	0.00164 &	0.000433 &	0 &	0.001044 &	0.000219 &	0 &	 \\
item-75 &	0.515971 &	0.001253 &	1010 &	708 &	53 &	9 &	6 &	1 &	0.008911 &	0.008475 &	0.018868 &	0.00092 &	0.001444 &	0.002976 &	 \\
user-75 &	0.515962 &	0.001336 &	1062 &	1559 &	70 &	1 &	10 &	1 &	0.000942 &	0.006414 &	0.014286 &	0.000751 &	0.0016 &	0.002232 &	 \\
user-10 &	0.580926 &	0.005989 &	3844 &	5281 &	430 &	15 &	47 &	11 &	0.003902 &	0.0089 &	0.025581 &	0.001671 &	0.004785 &	0.004144 &	 \\
user-40 &	0.558447 &	0.002253 &	2536 &	3620 &	230 &	4 &	19 &	1 &	0.001577 &	0.005249 &	0.004348 &	0.000589 &	0.001866 &	0.001064 &	 \\
avgs	 &	0.495028 &	0.001742 &	2574.888889 &	2488.444444 &	142.666667 &	8.666667 &	12.222222 &	1.666667 &	0.003614 &	0.004654 &	0.007468 &	0.000934 &	0.001458 &	0.001244 &	\\

%ItemBased - Timesplits 0 0 1 0 1
item-10 &	0.639232 &	0.000477 &	3307 &	2614 &	242 &	6 &	9 &	0 &	0.001814 &	0.003443 &	0 &	0.00041 &	0.000549 &	0 &	 \\
item-40 &	0.726871 &	0.01346 &	2271 &	1690 &	151 &	96 &	73 &	6 &	0.042272 &	0.043195 &	0.039735 &	0.012657 &	0.019136 &	0.010179 &	 \\
system-40 &	0.534059 &	0.003978 &	3048 &	2308 &	36 &	13 &	10 &	0 &	0.004265 &	0.004333 &	0 &	0.003523 &	0.001789 &	0 &	 \\
system-80 &	0.542178 &	0.001593 &	3048 &	2308 &	36 &	12 &	7 &	0 &	0.003937 &	0.003033 &	0 &	0.001467 &	0.001532 &	0 &	 \\
system-60 &	0.52987 &	0.004118 &	3048 &	2308 &	36 &	16 &	12 &	2 &	0.005249 &	0.005199 &	0.055556 &	0.003343 &	0.002371 &	0.005468 &	 \\
item-75 &	0.765866 &	0.037104 &	1010 &	708 &	53 &	97 &	74 &	4 &	0.09604 &	0.10452 &	0.075472 &	0.034926 &	0.036877 &	0.010268 &	 \\
user-75 &	0.718682 &	0.009978 &	1062 &	1559 &	70 &	22 &	63 &	4 &	0.020716 &	0.040411 &	0.057143 &	0.00273 &	0.013868 &	0.011675 &	 \\
user-10 &	0.658746 &	0.02711 &	3844 &	5281 &	430 &	68 &	129 &	29 &	0.01769 &	0.024427 &	0.067442 &	0.007861 &	0.018109 &	0.038802 &	 \\
user-40 &	0.689484 &	0.017514 &	2536 &	3620 &	230 &	55 &	97 &	14 &	0.021688 &	0.026796 &	0.06087 &	0.006168 &	0.014466 &	0.017258 &	 \\
avgs	 &	0.644999 &	0.012815 &	2574.888889 &	2488.444444 &	142.666667 &	42.777778 &	52.666667 &	6.555556 &	0.023741 &	0.028373 &	0.03958 &	0.00812 &	0.012078 &	0.010406 &	\\
\bottomrule\end{tabular}}\caption{asd]}\end{table}



\subsection{Does our proposed implicit rating methods improve the recommendation quality over binary preference data?}

Does our findings support our hypothesis?



\subsection{Compare the different implicit rating functions}

Does our findings support our hypothesis?

\subsection{Select the best combination of methods for the SoBazar recommender system}

Does our findings support our hypothesis?











%http://datacommunitydc.org/blog/2013/05/recommendation-engines-why-you-shouldnt-build-one/

Recommender systems are arguably one of the trendiest uses of data science startups today. However, with the exception of very
rare cases, it is not \emph{the killer} feature of your application which make users flock to you. The reason it works
well for Amazon and Netflix is because they have \emph{millions} of titles and a large existing user base. Presenting users
with recommended movies and products increase usage and sales, but does not create either to begin with.

The more data the better. With little or no data you won't be able to make recommendations \emph{at all}. Unless you have the users, domain expertise, algorithm development skill, massive inventory and frictionless user data entry your recommender
system will not be \emph{the milkshake that brings all the boys to the yard}. Instead the focus should be on building your core product, optimizing your e-commerce funnel, growing your user base, developing user loyalty and growing your inventory. In the meantime you can drive serendipitous recommendations simply by using a combination of most popular content and editors.


%TIMEBASED 80:20 BASELINES
%%ALSWR 20, 100, 5, true, 20
%Count linear		&	0.602177 &	0.003624 &	3038 &	2328 &	36 &	29 &	15 &	1 &	0.009546 &	0.006443 &	0.027778 &	0.002993 &	0.002128 &	0.004167 &	 \\
%Price linear		&	0.622406 &	0.004063 &	3038 &	2328 &	36 &	24 &	12 &	1 &	0.0079   &	0.005155 &	0.027778 &	0.003383 &	0.002566 &	0.041667 &	 \\
%Blend 1				&	0.607476 &	0.002832 &	3038 &	2328 &	36 &	19 &	11 &	2 &	0.006254 &	0.004725 &	0.055556 &	0.002206 &	0.003204 &	0.011619 &	 \\
%Popularity Linear	&	0.595155 &	0.001818 &	3038 &	2328 &	36 &	8  &	6  &	1 &	0.002633 &	0.002577 &	0.027778 &	0.000993 &	0.002204 &	0.013889 &	 \\
%Recentness linear	&	0.606503 &	0.004889 &	3038 &	2328 &	36 &	21 &	11 &	1 &	0.006912 &	0.004725 &	0.027778 &	0.004297 &	0.003416 &	0.001894 &	 \\
%%ALSWR 50, 100, 5, true, 20
%Count linear		&	0.582662 &	0.006386 &	3038 &	2328 &	36 &	17 &	21 &	2 &	0.005596 &	0.009021 &	0.055556 &	0.004606 &	0.008354 &	0.021465 &	 \\
%Price linear		&	0.583154 &	0.003985 &	3038 &	2328 &	36 &	18 &	10 &	3 &	0.005925 &	0.004296 &	0.083333 &	0.003159 &	0.003519 &	0.038131 &	 \\
%Blend 1				&	0.591816 &	0.004274 &	3038 &	2328 &	36 &	21 &	10 &	2 &	0.006912 &	0.004296 &	0.055556 &	0.003463 &	0.005744 &	0.007449 &	 \\
%Popularity Linear	&	0.591785 &	0.002544 &	3038 &	2328 &	36 &	14 &	11 &	1 &	0.004608 &	0.004725 &	0.027778 &	0.001385 &	0.004889 &	0.003472 &	 \\
%Recentness linear	&	0.599377 &	0.005037 &	3038 &	2328 &	36 &	23 &	16 &	2 &	0.007571 &	0.006873 &	0.055556 &	0.002428 &	0.018779 &	0.011742 &	 \\
%%ALSWR 100, 100, 5, true, 20
%Count linear		&	0.582662 &	0.006386 &	3038 &	2328 &	36 &	17 &	21 &	2 &	0.005596 &	0.009021 &	0.055556 &	0.004606 &	0.008354 &	0.021465 &	 \\
%Price linear		&	0.583154 &	0.003985 &	3038 &	2328 &	36 &	18 &	10 &	3 &	0.005925 &	0.004296 &	0.083333 &	0.003159 &	0.003519 &	0.038131 &	 \\
%Blend 1				&	0.591816 &	0.004274 &	3038 &	2328 &	36 &	21 &	10 &	2 &	0.006912 &	0.004296 &	0.055556 &	0.003463 &	0.005744 &	0.007449 &	 \\
%Popularity Linear	&	0.591785 &	0.002544 &	3038 &	2328 &	36 &	14 &	11 &	1 &	0.004608 &	0.004725 &	0.027778 &	0.001385 &	0.004889 &	0.003472 &	 \\
%Recentness linear	&	0.599377 &	0.005037 &	3038 &	2328 &	36 &	23 &	16 &	2 &	0.007571 &	0.006873 &	0.055556 &	0.002428 &	0.018779 &	0.011742 &	 \\
%%ALSWR 30, 0.065, 5, true, 20
%Count linear		&	0.574499 &	0.002075 &	3038 &	2328 &	36 &	20 &	10 &	0 &	0.006583 &	0.004296 &	0 		 &	0.001524 &	0.002217 &	0 		 &	 \\
%Price linear		&	0.549971 &	0.002668 &	3038 &	2328 &	36 &	19 &	9  &	0 &	0.006254 &	0.003866 &	0 		 &	0.001484 &	0.008032 &	0 		 &	 \\
%Blend 1				&	0.572434 &	0.003365 &	3038 &	2328 &	36 &	16 &	11 &	3 &	0.005267 &	0.004725 &	0.083333 &	0.00226  &	0.002841 &	0.036111 &	 \\
%Popularity Linear	&	0.560646 &	0.003052 &	3038 &	2328 &	36 &	13 &	10 &	0 &	0.004279 &	0.004296 &	0 		 &	0.002552 &	0.001974 &	0 		 &	 \\
%Recentness linear	&	0.557667 &	0.003213 &	3038 &	2328 &	36 &	15 &	14 &	1 &	0.004937 &	0.006014 &	0.027778 &	0.001418 &	0.010398 &	0.000947 &	 \\
%%ALSWR 30, 100, 10, true, 20
%Count linear		&	0.598733 &	0.002524 &	3038 &	2328 &	36 &	17 &	17 &	2 &	0.005596 &	0.007302 &	0.055556 &	0.001085 &	0.008913 &	0.01511  &	 \\
%Price linear		&	0.595108 &	0.0063 &	3038 &	2328 &	36 &	16 &	16 &	1 &	0.005267 &	0.006873 &	0.027778 &	0.005406 &	0.007655 &	0.041667 &	 \\
%Blend 1				&	0.593897 &	0.002231 &	3038 &	2328 &	36 &	14 &	12 &	3 &	0.004608 &	0.005155 &	0.083333 &	0.001114 &	0.007567 &	0.031061 &	 \\
%Popularity Linear	&	0.610599 &	0.001807 &	3038 &	2328 &	36 &	10 &	12 &	1 &	0.003292 &	0.005155 &	0.027778 &	0.000812 &	0.002884 &	0.013889 &	 \\
%Recentness linear	&	0.605114 &	0.002322 &	3038 &	2328 &	36 &	20 &	13 &	2 &	0.006583 &	0.005584 &	0.055556 &	0.002107 &	0.004453 &	0.010417 &	 \\

%Itembased - Random 0 0 1 0 1 
\begin{table}\centering\resizebox{\columnwidth}{!}{\begin{tabular}{*{19}l}\toprule
 & AUC &    MAP@20 &    T\_c &  T\_w &  T\_p &  P\_c &  P\_w &  P\_p &  R\_c &  R\_w &  R\_p &  MAP@20-click &  MAP@20-want &   MAP@20-purchase &    \\
\midrule
item-10 &   0.543288 &  0.001152 &  3293.5 &    2664 &  245 &   10.5 &  4 & 0.5 &   0.003188 &  0.001496 &  0.002033 &  0.001256 &  0.000223 &  0.000316 &   \\
system-40 & 0.620323 &  0.004098 &  2613.5 &    2494 &  284.5 & 28 &    31 &    8.5 &   0.010712 &  0.012422 &  0.029887 &  0.002901 &  0.004832 &  0.008935 &   \\
system-80 & 0.555757 &  0.002906 &  2613.5 &    2494 &  284.5 & 22 &    49.5 &  9.5 &   0.00843 &   0.019839 &  0.033347 &  0.001437 &  0.00458 &   0.006621 &   \\
user-75 &   0.698636 &  0.000566 &  1034.5 &    1307 &  104.5 & 1.5 &   3 & 0.5 &   0.001296 &  0.002296 &  0.004167 &  0.00095 &   0.000512 &  0.000204 &   \\
user-40 &   0.7072 &    0.002332 &  2472.5 &    3111 &  215 &   10.5 &  17.5 &  3.5 &   0.004416 &  0.005627 &  0.014714 &  0.001559 &  0.001528 &  0.001492 &   \\
item-75 &   0.721677 &  0.0008 &    930 &   784 &   68.5 &  4.5 &   2.5 &   0 & 0.004799 &  0.003187 &  0 & 0.000927 &  0.000974 &  0 &  \\
system-60 & 0.570296 &  0.003737 &  2613.5 &    2494 &  284.5 & 29.5 &  51 &    10.5 &  0.011287 &  0.020453 &  0.036843 &  0.002571 &  0.003853 &  0.009554 &   \\
item-40 &   0.763909 &  0.001451 &  2198.5 &    1769 &  170.5 & 21 &    11 &    2.5 &   0.009565 &  0.006221 &  0.014553 &  0.001477 &  0.001053 &  0.001598 &   \\
user-10 &   0.606411 &  0.010553 &  3707 &  4633 &  331.5 & 47 &    56 &    8 & 0.012926 &  0.012057 &  0.023397 &  0.00467 &   0.007241 &  0.003836 &   \\
avgs     &  0.643055 &  0.003066 &  2386.277778 &   2416.666667 &   220.944444 &    19.388889 & 25.055556 & 4.833333 &  0.007402 &  0.009289 &  0.01766 &   0.001972 &  0.002755 &  0.003617 &  \\
\bottomrule\end{tabular}}\caption{cold mahout-itembased}\end{table}
%SVD - Random 0 0 1 0 1
\begin{table}\centering\resizebox{\columnwidth}{!}{\begin{tabular}{*{19}l}\toprule
 & AUC &    MAP@20 &    T\_c &  T\_w &  T\_p &  P\_c &  P\_w &  P\_p &  R\_c &  R\_w &  R\_p &  MAP@20-click &  MAP@20-want &   MAP@20-purchase &    \\
\midrule
item-10 &   0.636927 &  0.000058 &  3293.5 &    2664 &  245 &   2 & 2.5 &   0.5 &   0.000608 &  0.000939 &  0.002049 &  0.000032 &  0.000183 &  0.000199 &   \\
system-40 & 0.738839 &  0.015613 &  2613.5 &    2494 &  284.5 & 96 &    124.5 & 17 &    0.036705 &  0.049898 &  0.059644 &  0.011173 &  0.020444 &  0.017703 &   \\
system-80 & 0.677093 &  0.009682 &  2613.5 &    2494 &  284.5 & 61.5 &  77.5 &  11 &    0.023539 &  0.031049 &  0.038702 &  0.007366 &  0.00979 &   0.007739 &   \\
user-40 &   0.808431 &  0.043284 &  2472.5 &    3111 &  215 &   113 &   199 &   11.5 &  0.04532 &   0.063955 &  0.052234 &  0.018144 &  0.037683 &  0.013304 &   \\
user-75 &   0.833505 &  0.014064 &  1034.5 &    1307 &  104.5 & 31.5 &  62 &    5 & 0.029431 &  0.047435 &  0.047472 &  0.008494 &  0.017224 &  0.008656 &   \\
item-75 &   0.789062 &  0.045703 &  930 &   784 &   68.5 &  113 &   100.5 & 9 & 0.121531 &  0.128217 &  0.138092 &  0.037182 &  0.054105 &  0.041008 &   \\
system-60 & 0.6974 &    0.010945 &  2613.5 &    2494 &  284.5 & 71.5 &  80.5 &  9.5 &   0.027375 &  0.032255 &  0.033384 &  0.008655 &  0.013835 &  0.015957 &   \\
item-40 &   0.817333 &  0.022964 &  2198.5 &    1769 &  170.5 & 155 &   150 &   11.5 &  0.070489 &  0.08479 &   0.06747 &   0.017619 &  0.031348 &  0.022649 &   \\
user-10 &   0.728219 &  0.04259 &   3707 &  4633 &  331.5 & 134 &   180 &   13.5 &  0.036112 &  0.038847 &  0.041103 &  0.019312 &  0.029633 &  0.0155 &     \\
avgs     &  0.747423 &  0.022767 &  2386.277778 &   2416.666667 &   220.944444 &    86.388889 & 108.5 & 9.833333 &  0.043457 &  0.053043 &  0.05335 &   0.01422 &   0.023805 &  0.015857 &  \\
\bottomrule\end{tabular}}\caption{cold mahout-svd}\end{table}





\section{Discussion}

In Chapter 1 we introduced a total of 8 goals for our Master's thesis:

\begin{itemize}
\item G1: Gain a better understanding of the fashion domain.
\item G2: Identify the specific challenges of making fashion recommendations.
\item G3: Study how existing technologies can be adapted to mitigate or
  		  overcome these challenges.
\item G4: Study existing solutions to the cold-start.
\item G5: Identify the best suited methods with regard to both application and domain.
\item G6: Explore the existing solutions of how to infer user preference from implicit feedback data.
\item G7: Establish user interaction patterns to support our assumptions.
\item G8: Find different methods of combining various event types into implicit ratings.
\item G9: Find metrics in order to evaluate the \emph{implicit ratings}
\end{itemize}

In the following subsection we will reiterate these goals and discuss whether we have succeeded in reaching our goals. Readers should
note that we already have performed a detailed analysis of the results of the work on goals G1 in Section \ref{sec:dataset-conclusion},
G2 in Section \ref{}, G3 in Section \ref{}, G4 and G5 in Section \ref{sec:cold-start-discussion}, G6 in Section \ref{implicit-weaknesses},
G7 in Section \ref{}, G8 in Section \ref{} and finally G9 in Section \ref{sec:evaluation-metrics}.
The following discussion will therefore be fairly high-level, and avoid the \emph{nitty-gritty} details.

\subsection{G1, G2 and G3: Solutions to the fashion domain related problems}
\label{sec:fashion-discussion}

Through thorough examination and analysis of the data and a literature review of fashion articles we came to the following main conclusions: sparse data should be supplemented with other data to improve results and the variety of ways to do so, time is central and fashion is built on subcultures.

After analyzing the data it became apparent that the data was not just sparse in the sense of users, but also in the sense of items and item interactions.
The sparsity of the data
    Few users
    Few interactions on items
    Few purchases per user
        Naturally more clicks and wants
            Higher rate of purchases based on more clicks
                Can and should use this to supplement the low purchase data with click and want data

Time
    Many user accounts are shortlived
    Many items are shortlived
        Must handle items as decaying entities
            Recommendations must be supplemented with the recentness of the data

Fashion
    Fashion not only to wear something - Fashion for showing off
        Expect to find evidence of user preferences in content data
    Beloning to a subculture
        Expect to find user preferences in user demographics


%TODO - Dicuss data findings

%Important notes

\subsection{G4: The Cold-Start Problem}
\label{sec:cold-start-discussion}

Through our literature review, we closely examined five different classes of solutions to the cold-start problem. Namely trust-aware recommender system, filterbots, seed users, interview process and hybrid methods.

However, a question that should be asked is whether there exists and solutions which we did not discover or was excluded from our review. An issue that could have influenced the results of our review is \emph{researchers bias}, which could have arisen of any of the researchers had preferences for one of the solution types before the literature search. The reason for including multiple methods that never were used was due to uncertainty regarding what data would become available during the project. Trust-aware recommender system and hybrid methods in particular could never be tested out due to a lack of social/demographic data, which we were hoping to get access to.



After closely examining the different methods we concluded that both filterbots and content-boosted hybrid methods could provide a possible solution for
our specific scenario...

Demographic information could further improve its performance and usefulness. Agarwal et. al. \cite{Agarwal2009} used 13 filterbots in their experiments, where 11 out of 13 bots rated items based on demographic information.


\subsection{G5, G6, G7: Implicit Ratings}
\label{sec:implicit-discussion}

Importance of implicit factors for the domain

Another question that arises is: when does it become obsolete to look at clicks and wants?

%TODO - Discuss implicit ratings





\section{Issues}\label{sec:issues}

%TODO - Mention that the recommender system libraries are terribad for evaluation...
