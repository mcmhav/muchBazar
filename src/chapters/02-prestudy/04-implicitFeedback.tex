% !TEX root = ../../report.tex

% What will be adressed in the section?
  % Motivation for using implicit feedback

	% Implicit feedback vs. explicit feedback
		% How is implicit feedback related to user preference

  % Evaluating implicit ratings, extracted from implicit feedback.

  % Closely related article:
    % Evaluation -> And specifically, evaluation of implicit ratings and
    % predictions based on these.

\section{Implicit feedback to implicit ratings}

Explicit feedback is present in many of the largest recommender systems today
and hence, is extensively researched.\todo{Ref.} The user is commonly asked to
rate item $i$ on a Likert scale from $1$ to $k$, ranging from strongly
disagree/dislike to strongly agree/like. Its advantages are, among others the
ability to get precise feedback from the user and capturing both positive and
negative preferences. However, although having a high popularity, the method
has multiple weaknesses. The most prominent weakness is the difficulty of
collecting ratings: the method requires the user to spend time rating items and
the amount of feedback is often scarce, creating sparse data sets. Further,
explicit ratings are often subject to inconsitencies known as natural
noise \cite{amatriain2009like} and users might also be pressed to report
different preferences due to peer of social pressure. The fact that we are
introducing a user overhead, makes it difficult to have a complete view on the
user preferences \cite{jawaheer2010characterisation}.

We can achieve better results looking at user behaviour, which is both easier
to collect and does not require any extra effort from the user.  Our end-goal
is to predict a rating $r$ for a given user $u$ on item $i$, and thus we need
some way of translating our implicit feedback into what we call
\textbf{implicit ratings} - these are ratings, just like explicit ratings, but
inferred by user behaviour and as we will see in the succeeding sections need
to be anylyzed with this in mind.

\subsection{Quantifying implicit feedback}

Hu et. al. \cite{Hu2008} identify four unique characteristics of implicit
feedback, which differentiates it from explicit feedback...

\begin{enumerate}

\item No negative feedback. By observing user behaviour we can infer which
items the user consume and probably like. However, it is hard to infer which
items the user did not like. This asymmetry has several implications; Explicit
feedback provides a more detailed picture of the
users preferences, but for implicit data the low ratings are treated as missing
data and omitted from the analysis. Hence it is crucial to address the missing
data where most negative feedback is expected to be found

\item Implicit feedback is inherently noisy. While we track user behaviour, we
can only guess their preferences and true motives. For example, a purchase does
not necessarily indicate a positive view of an item, the item may have been
purchased as a gift, or perhaps the user was disappointed
with the item

\item The numerical value of explicit feedback indicates preference, whereas
the numerical value of implicit feedback indicates confidence. Explicit
feedback could e.g. range from total dislike to really like, on the other hand
implicit feedback describe the frequency of actions, e.g. how
frequently a user buys an item. But a higher frequency might not necessarily
indicate a stronger preference. A user might choose to only watch a really good
movie once. However, a recurring event is more likely to reflect the user
opinion. However, the numerical value of the feedback is definitely useful, as
it tells us about the confidence we have in a certain observation

\item Evaluation of implicit feedback requires appropriate measures. In the
case of explicit feedback where a user specify a numerical score, measures such
as mean squared error (MSE) could measure the success of the predictions.
However, with implicit models we have to take into account
the availability of the item, competition with other items, and repeat
feedback.

\end{enumerate}

\subsection{Challanges and weaknesses}

Almost all of the research on implicit feedback has considered how behaviors can
be used as positive evidence, rather than negative evidence. However, one can imagine
behaviors which indicate that a user does not find something relevant or which suggest
that something is unimportant to the user, such as delete. It is likely that little research
has been conducted on negative implicit feedback because there are fewer of these types
of behaviors, and, in general, less is understood about how to effectively use negative
feedback, whether for implicit or explicit relevance feedback.

Another challenge facing implicit feedback research is the notion of degree of
personalization offered by the system. In particular, individual differences can greatly
impact the effectiveness of using behavior as implicit relevance feedback. People behave
differently and have varying approaches to information-seeking; thus, it is difficult to
generate, and dangerous to apply, all-purpose rules for describing how behavior can be
used as implicit relevance feedback.

\subsection{Evaluating convertion to implicit ratings}

